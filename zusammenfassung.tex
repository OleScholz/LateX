\chapter{Zusammenfassung und Ausblick}
Das Ziel dieser Arbeit war der Entwurf einer Grid Fin Aktuatorik für wiederverwendbare AirLaunch-Raketen. Hierfür wurden nach eine kurzen Einleitung in das Thema die technischen Grundlagen beschrieben. Erst wurden die wichtigsten Begrifflichkeiten geklärt, um die Grid Fins und ihre Orientierung beschreiben zu können. Anschließend konnte detailliert auf ihr strömungsmechanisches Verhalten, bisherige Verwendungen sowie das AirLaunch-System Valkyrie, für das die Grid Fins entworfen wurden, eingegangen werden.

In Kapitel \ref{sec:modellentwurf} wurden dann aus einer Betriebssimulation der Valkyrie Bedingungen wie zum Beispiel die Strömungszustände und wirkende Kräfte ausgelesen, die ausschlaggebend für die Belastungen des zu entwerfenden Systems waren. Die Simulation wurde ohne ReEntry-Burn und mit angestellten Grid Fins durchgeführt, um bei der Auslegung auch extreme Bedingungen zu berücksichtigen. Weitere Bedingungen ergaben sich aus dem Fertigung im 3D-Druck.

Anschließend wurden zwei morphologische Kästen erstellt. In denen jeweils die Designoptionen für die Grid Fins oder Aktuatorik dargestellt wurden. Die Ansätze aus dem Grundlagenkapitel konnten hier angewandt werden, sodass sich für die Grid Fins fünf Kategorien (Gitterform, Zellform, Wandquerschnittform, Krümmung und Pfeilung) und für die Aktuatorik drei Hybriken (Art des Aktuators, des Getriebes und der Lagerung zur Wahl) ergaben.

Im Folgenden wurde dann unter Berücksichtigung der vorher definierten Anforderungen ein erstes Systemdesign entworfen. Zunächst Edelstahl 1.4542 als Werkstoff festgelegt. Als nächstes wurde der Grid Fin als rechteckige Rahmenform und einer konkaven Krümmung zur Anströmung hin definiert. Die Zellen wurden quadratisch entworfen, ihre Zellwände auf beiden Seiten zugespitzt und außerdem mit einer lokalen Pfeilung versehen. Anschließend konnten die zwei Bewegungen, das Ausklappen vor dem Wiedereintritt und das Steuern während der atmosphärischen Flugphase, getrennt betrachtet werden. Dennoch ergab für beide ähnliche Lösungsansätze mit Elektromotoren und Wälzlagern. Für den Ausklappmechanismus wurde jedoch die rotarische Bewegung des Motors über ein Spindelgetriebe in eine Linearbewegung  umgewandelt, die den benötigten Hub für die Klappbewegung bewirkt. Diese Klappaktuatorik wurde auf der Welle der Steueraktuatorik montiert, bei der stattdessen ein Planetengetriebe verwendet wurde. 

Die Kräfte aus der Simulation wurden dann in FEM-Berechnungen auf die Bauteile aufgebracht. Somit konnten die Bauteile hinsichtlich der Belastungen optimiert werden. Dabei hat es sich überraschender Weise herausgestellt, dass die Belastungen im gesamten Gitter recht gleichmäßig verteilt sind, anstatt -wie zuerst angenommen- zur Einspannung hin zu steigen. Deshalb wurde eine konstanten Wandstärke von $1,5$ mm implementiert. Des Weiteren wurde festgestellt, dass besonders bei der Welle durch eine Reduktion des Volumens viel Masse eingespart werden konnte.

Neben FEM-Berechnungen wurden auch zwei Betriebssimulationen für die beiden Manöver durchgeführt. Sie haben ergeben, dass die Bewegungen anforderungsgerecht bewältigt werden können und besonders die Klappaktuatorik selbst unter erschwerten Bedingungen gute Ergebnisse liefert. 
\\~\\
In Zukunft muss aber noch die thermische Belastbarkeit Grid Fins unter den Wiedereintrittsbedingungen bestimmt werden. Hierfür sollte idealer Weise eine CFD-Simulation durchgeführt werden, aus der sich neben thermischen Lasten auch die genauen Widerstands und Auftriebsbeiwerte bestimmen lassen. Für diese könnte aber auch alternativ ein Windtunneltest durchgeführt werden. Um ein realistisches Bild der aerodynamischen Kräfte zu erhalten, sollten dabei auch Wechselwirkungen mit dem Raketenkörper berücksichtigt werden.

Eine Möglichkeit der Gewichtsreduzierung bietet das Weglassen der Klappaktuatorik. Dies spart gleichzeitig Kosten und Baumraum in der Rakete. Des Weiteren wird die Rakete dadurch simpler und es gibt weniger bewegte Teile die zu Problemen führen könnten. Damit Grid Fins ohne Klappbewegung auch wirklich rentabel sind, muss geprüft werden, welche Folgen dies auf den Start bezüglich Widerstand und Stabilität hat.

Auch die Art der Fertigung kann noch ein mal neu evaluiert werden. Mit einem 3D-Druck-Preis von 12.900€ ist ein Grid Fin nicht gerade günstig und Fräsen könnte sich als günstiger herausstellen. Wird eine starke Vereinfachung der Geometrie des Grid Fins in Kauf genommen, können sogar deutlich günstigere Verfahren wie zum Beispiel das zusammenschweißen von Blechen geprüft werden, da die Wanddicke ohnehin konstant ist.

Um das Potenzial der Grid Fin maximal nutzen zu können, müssten des Weiteren auch noch die Regler für den Klapp- und Steuerwinkel optimiert werden.