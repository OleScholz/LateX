\chapter{Zusammenfassung und Ausblick}
Das Ziel dieser Arbeit war der Entwurf einer Grid Fin Aktuatorik für wiederverwendbare AirLaunch-Raketen. Hierfür wurden nach eine kurzen Einleitung in das Thema die technischen Grundlagen beschrieben. Erst wurden die wichtigsten Begrifflichkeiten geklärt, um die Grid Fins und ihre Orientierung beschreiben zu können. Anschließend konnte detailliert auf ihr strömungsmechanisches Verhalten eingegangen werden. Hierfür konnte eine Vielzahl von Studien zur Hilfe herangezogen werden, die ihre Aerodynamik beschreiben und somit ein Verständnis bieten, wie sich ihre Flugeigenschaften manipulieren lassen. Schließlich wurde noch auf bisherige Verwendungen von Grid Fins eingegangen und das AirLaunch-System Valkyrie beschrieben, für das die Grid Fins entworfen wurden.

In Kapitel \ref{sec:modellentwurf} wurden dann aus den Betriebsbedingungen der Valkyrie und weiteren Angaben aus der Aufgabenstellung Anforderungen formuliert. Aus einer Betriebssimulation ließen sich zum Beispiel die Strömungsbedingungen und wirkende Kräfte auslesen, die ausschlaggebend für die Belastungen des zu entwerfenden Systems waren. Die Simulation wurde nicht nur für den Normalfall, sondern auch ohne ReEntry-Burn und mit angestellten Grid Fins, durchgeführt. Somit wurde bei der Auslegung darauf geachtet, dass es auch unter diesen erschwerten Bedingungen nicht zum Versagen kommt. Eine zentrale Forderung in der Aufgabenstellung war es, die Grid Fins für eine Fertigung im 3D-Druckverfahren auszulegen. Neben der Einschränkung der Größe durch den verfügbaren Bauraum hat dies den Vorteil, dass auch komplexe Strukturen mit geringem Aufwand fertigbar sind.

Anschließend wurden zwei morphologische Kästen erstellt. In einem wurde die Designoptionen für die Grid Fins übersichtlich dargestellt. Hier konnten die verschiedenen Ansätze, die schon im Grundlagenkapitel angesprochen wurden, aufgegriffen und gegeneinander abgewogen werden. Es wurde zwischen fünf Kategorien unterschieden: Gitterform, Zellform, Wandquerschnittform, Krümmung und Pfeilung. Der zweite morphologische Kasten bot eine Übersicht über die verschieden Möglichkeiten die Aktuatorik zu gestalten. So standen die Art des Aktuators, des Getriebes und der Lagerung zur Wahl.

Im Folgenden wurde dann unter Berücksichtigung der vorher definierten Anforderungen ein erstes Systemdesign festgelegt. Zunächst wurden verschiedene Materialien, die für den 3D-Druck zur Verfügung stehen gegeneinander abgewogen und Edelstahl als Werkstoff festgelegt. Als nächstes wurde der Grid Fin als rechteckige Rahmenform mit einem Seitenverhältnis von 5:6 und einer konkaven Krümmung zur Anströmung hin definiert. Die Zellen wurden quadratisch entworfen und ihre Zellwände auf beiden Seiten zugespitzt und außerdem mit einer lokalen Pfeilung versehen. Dann wurde der entsprechende Aufbau als CAD-Modell implementiert. Anschließend konnte die zugehörige Aktuatorik entworfen werden. Auch wenn die zwei Bewegungen, das Ausklappen vor dem Wiedereintritt und das Steuern während der atmosphärischen Flugphase getrennt betrachtet wurden, haben sich für beide ähnliche Lösungsansätze herausgebildet. Beide Male wurden Elektromotoren und Wälzlager verwendet. Für den Ausklappmechanismus wird die rotarische Bewegung des Motors über ein Spindelgetriebe in eine Linearbewegung  umgewandelt, die den benötigten Hub für diese Manöver liefert. Diese Klappaktuatorik wurde auf eine Welle montiert, über die die Steuerbewegung ausgeführt wird. Diese hat schon die richtige Bewegungsart und musste deswegen nur noch über Planetengetriebe verstärkt werden. Während beim Grid Fin schon in diesem Kapitel genaue Maße genannt wurden, ergibt dies für die Aktuatorik noch wenig Sinn, da sie stark von noch änderungsanfälligen Werten abhängen.

Somit stand der erste Entwurf fest, der nun überprüft und optimiert werden konnte. Hierzu wurden zunächst die Kräfte, die sich aus der Betriebssimulation ergeben hatten. auf den Grid Fin in einer FEM-Simulation aufgebracht. Es wurde zuerst die Halterung und dann das Gitter angebracht. Dabei hat es sich überraschender Weise herausgestellt, dass die Belastungen im gesamten Gitter recht gleichmäßig verteilt sind, anstatt -wie zuerst angenommen- zur Einspannung hin zu steigen. Dadurch war die Implementierung einer konstanten Wandstärke möglich. Diese wurde aber nicht bis auf ein Erreichen von kritischen Spannungen unter Rücksichtnahme eines Sicherheitsfaktors reduziert, sondern auf einem festen Wert von $1,5$mm gelassen, da die thermischen Lasten unbekannt sind und diese vermutlich eher der entscheidende Faktor sein könnten. Anschließend konnte der Rest der Aktuatorik ebenfalls überprüft und genau festgelegt werden. So wurden zum einen an dieser Stelle die Lager aus Katalogen von verschieden Herstellern gewählt und zum anderen auch die Maße der weiteren Bauteile festgelegt. Zum Überprüfen der einzelnen Komponenten wurden dann sowohl numerische als auch analytische Rechnungen durchgeführt, die die Belastbarkeit der Bauteile bestätigt und noch weitere Anpassungen ermöglicht haben. Besonders an der Welle hat es sich herausgestellt, dass sie aufgrund der Kinematik deutlich dicker ausgelegt werden musste als aus strukturmechanischer Sicht notwendig. Somit konnte an ihr noch viel Material durch zum Beispiel Aushöhlung entfernt werden, sodass  die Masse deutlich reduziert werden konnte.

Neben der mechanischen Auslegung wurde auch noch die Leistung der gewählten Aktuatoren überprüft. In zwei separaten Betriebssimulationen wurde sowohl das Manöver des Ausklappens als auch die Steuerung getestet. Der Klappaktuator hat es geschafft, den Grid Fin innerhalb von einem Bruchteil der Zeit, die ihm maximal dazu zur Verfügung stände, auszuklappen. Auch der Steueraktuator wurde bestätigt, indem er es schaffte, den Grid Fin bei maximaler Belastung in der vorher definierten Zeit von der Neutralstellung zum maximalen Ausschlag in beide Richtungen und wieder zurück  zu erreichen. Somit erfüllt das System alle definierten Anforderungen.
\\~\\
Bevor dieses System jedoch verwendet werden kann, müssten noch weitere Dinge, die nicht in dieser Arbeit behandelt wurden, bestätigt werden. Der wichtigste Aspekt ist hier wohl die thermische Last, die beim Wiedereintritt auf die Grid Fins einwirkt. Das Standhalten der mechanischen Belastung ist nur gewährleistet, wenn das Material durch die Wärme nicht beginnt weich zu werden oder gar zu schmelzen. Da dies ein kompliziertes Phänomen ist, sollten in Zukunft noch Simulationen zu diesem Thema durchgeführt werden. In Rahmen dieser Simulationen kann dann auch die Aerodynamik neu evaluiert werden. Die bisherigen Auftriebs- und Widerstandsbeiwerte sind nur Näherungen, die aus bekannten Werten abgeleitet wurden. Genauere Werte müssten noch mit einer CFD-Simulation ermittelt werden. Alternativ könnte auch ein Modell im Windtunnel getestet werden, um daraus die Beiwerte abzuleiten. Herbei sollt nicht nur der Grid Fin alleine betrachtet werden, sondern auch die Wechselwirkungen mit dem Raketenkörper in Betracht gezogen werden. Hierbei sei bedacht, dass eine erneute Anpassung der Dimensionen der  Aktuatorik von Nöten sein könnte, falls die Wandstärke der Gitterwände aufgrund der thermischen Analysen erhöht werden.

Diese Anpassungen der Aktuatorik werfen die Frage auf, ob ein Weglassen der Klappbewegung nicht vorteilhafter sein könnte. Die Welle nimmt sehr viel Bauraum innerhalb der Rakete ein und trägt signifikant zur Gesamtmasse des Systems bei. Da der Klappaktuator auf ihr befestigt wird, muss sie eine recht große Länge besitzen. Fällt dieser jedoch weg, kann sie deutlich reduziert werden. Auch die Kosten und Fehleranfälligkeit gehen dadurch deutlich herunter. Ein Nachteil ist jedoch, dass die Grid Fins in dem Fall auch schon beim Start ausgeklappt wären und somit verstärkt zum Widerstand beitragen. Es wäre jedoch möglich den Grid Fin so zu drehen, dass er sowohl beim Start als auch dem Wiedereintritt mit der Pfeilung zur Anströmung zeigt. Damit ließe sich der erzeugte Widerstand leicht verringern. Auch die Flugstabilität könnte beim Aufstieg dadurch gefährdet werden, was durch eine Steuerung der Grid Fins nicht zwangsläufig behebbar ist.

Auch die Art der Fertigung kann noch einmal neu evaluiert werden. Bisher wurde von einem 3D-Druckverfahren ausgegangen, weswegen teilweise komplizierte geschwungene Strukturen konstruiert werden konnten. Jedoch ist selbst beim 3D-Druck die Darstellung von zum Beispiel den spitz zulaufenden Wänden nur in Abhängigkeit von der Auflösung des endgültig gewählten Druckers möglich. Es kann auch gut sein, dass die thermischen Analysen ergeben, dass diese spitze Kanten ohnehin nicht für diese Anwendung ratsam wären. Somit sollte geprüft werden, ob nicht doch ein anderes Fertigungsverfahren, wie zum Beispiel Fräsen, günstiger ist. Bei konstanter Wandstärke könnte sogar noch simplere Verfahren geprüft werden.

Die Grid Fins bräuchten natürlich noch einen Regler für den Steuerwinkel, um im Flug ihre maximale Effektivität zu erreichen. Dafür sollte eventuell eine Kaskadenreglung gewählt werden, die unter Berücksichtigung der Rotationsgeschwindigkeit $\dot{\delta}$ noch bessere Ergebnisse liefert. Mit einem ordentlich funktionierenden Regler kann dann überprüft werden, wie gut die Grid Fin sich wirklich eignen in der Atmosphäre zielgenau dem Helikopter entgegen zu fliegen oder einfach nur lange genug zu gleiten, um Zeit für die Bergung zu gewinnen.