\chapter*{Übersicht}
Grid Fins, als Möglichkeit Raketen beim Wiedereintritt zu lenken, haben in den letzten Jahren mit steigender Nachfrage der NewSpace-Branche immer mehr Aufmerksamkeit erhalten. Somit war es das Ziel dieser Arbeit eine Grid Fin Aktuatorik für wiederverwendbare Raketen zu entwerfen. Mit Hilfe von zwei morphologischen Kästen wurden verschiedene Designansätze gegeneinander abgewogen und überprüft, welche den Anforderungen gerecht werden. Die Anforderungen ergaben sich aus der gewünschten Art der Fertigung, dem 3D-Druck, und einer Betriebssimulation einer Raketenmission unter erschwerten Bedingungen, wie zum Beispiel dem Wiedereintritt ohne ReEntry-Burn.  Eine erste Version wurde anschließend in CAD modelliert und erste Komponenten für die Aktuatorik aus Katalogen gewählt. Das Design wurde dann mittels FEM-Berechnungen untersucht. Dabei hat es sich ergeben, dass die Belastungen im Grid Fin deutlich geringer sind als erwartet und es auch keinen signifikanten Anstieg der Spannungen zur Anspannung hin ergibt. Auf Grund dieser FEM-Ergebnisse konnte die Wandstärke auf einen konstanten Wert reduziert werden. Die Belastungen in den restlichen Bauteilen wurde mit einer Mischung aus numerischen und analytischen Rechnungen ebenfalls untersucht und angepasst. Der Grid Fin sollte sich außerdem um zwei Achsen bewegen lassen. Die gewählte Aktuatorik für die Steuerbewegung, bestehend aus Elektromotor, Wälzenkörperlagerung und Planetengetriebe, wurde in einer Betriebssimulation erfolgreich auf ihre Fähigkeit getestet, entgegen das durch die Aerodynamik erzeugte Moment in begrenzter Zeit die geforderte Bewegung auszuführen. Auch die Aktuatorik, die im Gegensatz dazu ein Spindelgetriebe verwendet und zum Ausklappen der Grid Fins nach der Trennung von der 2. Stufe führt, erfüllte in einer Simulation die Vorgabe, vor Beginn des Wiedereintritts ihr Manöver zu beenden.