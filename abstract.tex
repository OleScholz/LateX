\chapter*{Übersicht}
Im Rahmen des NewSpace finden Gitterflossen, auch Grid Fins genannt, immer häufiger Anwendung als Möglichkeit zur aktiven Steuerung von Wiedereintrittsfahrzeugen und wiederverwendbaren Raketenstufen. Da nun auch neue Microlauncher und AirLaunch-Raketen wiederverwendbar werden, kann auch für sie eine Nutzung von Grid Fins in Betracht gezogen werden. Somit ist es das Ziel dieser Arbeit eine Grid Fin Aktuatorik für solche Raketen zu entwerfen. Mit Hilfe von zwei morphologischen Kästen werden verschiedene Designansätze gegeneinander abgewogen und überprüft, welche den Anforderungen gerecht werden. Die Anforderungen ergaben sich aus der gewünschten Art der Fertigung, dem 3D-Druck, und einer Betriebssimulation einer Raketenmission unter erschwerten Bedingungen, wie zum Beispiel dem Wiedereintritt ohne ReEntry-Burn.  Eine erste Version wird anschließend in CAD modelliert und auch erste Komponenten für die Aktuatorik aus Katalogen gewählt. Das Design wird dann mittels FEM-Berechnungen untersucht. Dabei wird zunächst versucht die Verteilung und der Fluss der Spannung im Material zu verstehen und dem zufolge ein massearmes und gleichzeitig spannungsgerechtes Design zu erreichen. Die Belastungen in den restlichen Bauteilen wurde mit einer Mischung aus numerischen und analytischen Rechnungen ebenfalls untersucht und angepasst. Der Grid Fin sollte sich außerdem um zwei Achsen bewegen lassen. Die gewählte Aktuatorik für die Steuerbewegung, bestehend aus Elektromotor, Wälzenkörperlagerung und Planetengetriebe, wird in einer Betriebssimulation auf ihre Fähigkeit getestet. Gleiches gilt für die Aktuatorik zum Ausklappen er Grid Fins, welche im Gegensatz dazu ein Spindelgetriebe verwendet. Am Ende konnte gezeigt werden, dass ein anforderungsgerechtes Design für die Grid Fin Aktuatorik zustande kam.