\chapter{Einleitung}

Die Einleitung soll einen Überblick über den Stand der Technik geben, das zu unter-
suchende System beschreiben und die Aufgabenstellung mit eigenen Worten näher
erläutern.

\section{Motivation}

\section{Ziele der Arbeit}

\section{Vorgehensweise}
Im nächsten Kapitel werden zunächst die für diese Arbeit notwendigen Grundlagen dargelegt. Zu Beginn wird auf die Eigenschaften von Grid Fins eingegangen, sowohl in Bezug auf ihr aerodynamisches Verhalten, als auch unter Betrachtung ihrer Vor- und Nachteile gegenüber konventionellen planaren Steuerflächen. Als nächstes werden dann die Wiedereintrittsbedingungen bei einer suborbitalen Flugbahn am Beispiel des Airlaunchsystems Valkyrie erläutert. Es folgt ein kurzer Einschub zur Aktuatorik mit der die Grid Fins gesteuert werden könnten.

Nachdem die Grundlagen geklärt sind, werden in Kapitel \ref{sec:modellentwurf} die Anforderungen an das System definiert. Unter Berücksichtigung dieser folgt eine Vorstellung verschiedener Teillösungen für die einzelnen Elemente von Steuerflächen und Aktuatorik. Auf Basis eines Morphologischen Kastens, in dem diese Teillösungen zusammengetragen werden, wird begründet ein erster Demonstrator entworfen und in einem CAD-Programm erstellt.

Daraufhin wird dieses Modell in Kapitel \ref{sec:simulation} mittels einer Finiten Elementen Berechnung auf Stabilität und Festigkeit untersucht und mit einer Betriebssimulation in Matlab auf eine genügende
Leitungsfähigkeit im Betrieb geprüft. Auf Grund dieser Simulationen wird das Modell verbessert und anschließend kritisch bewertet.
Zuletzt werden noch einmal alle Ergebnisse zusammengefasst und ein Ausblick auf mögliche weitere Arbeitsschritte gegeben.