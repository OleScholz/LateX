\chapter{Materialeigenschaften}
Abbildung \ref{abb_100mmModel} zeigt das simple Modell, dass zum Vergleich der Materialkosten genutzt wurde. Es wurde mit einem Volumen von ungefähr $100$ mm$^3$ entworfen, was ein überschlägiges Umrechnen in andere Volumina ermöglicht. Die Gitterstruktur und die Krümmung wurden ebenfalls implementiert, um die wesentlichen Charakteristika eines Grid Fins, die große Auswirkungen auf den Preis haben könnten und mit wenig Aufwand einzubringen sind, zu berücksichtigen.
\begin{figure}[h]
	\centering
	\includegraphics[width=0.5\textwidth]{PreisModell.png}
	\caption{Modell zur Ermitllung der Materialkosten}
	\label{abb_100mmModel}
\end{figure}
\begin{itemize}
		\item $V = 99,70\mathrm{mm}^3$
		\item Benötigter Bauraum: 187,8 x 132,3 x 44,2 mm$^3$
		\item Höhe und Breite des Gitters $b = h = 132,3$mm
		\item Sehnenlänge $s = 14,2$mm
		\item Wanddicke am Rahmen $d_R = 3,8$mm
		\item Wanddicke im Gitter $d_G = 1,9$mm
		\item Krümmungsradius $=200$mm
\end{itemize}
Tabelle \ref{tab_Werkstoffe} zeigt einen Vergleich der für den 3D-Druck gängigen Werkstoffe, die auch für die Raumfahrt in Frage kommen.
\begin{landscape}
\begin{table}[h]
	\centering
	\caption{Vergleichsdaten der unterschiedlichen Werkstoffe}
	\label{tab_Werkstoffe} 
	\begin{tabular}{|c|c|c|c|c|c|c|c|c|}
		\hline
		Werkstoff&Bezeichnung&$\rho$/$\frac{\mathrm{g}}{\mathrm{cm}^3}$&$R_{p,0.2}$/MPa&$R_{p,0.2}$/MPa&$R_\mathrm{spez.}$/$\frac{\mathrm{Nm}}{\mathrm{g}}$&Preis/€&$T_\mathrm{E, max}$/$^\circ$C&$T_\mathrm{Schmelz}$/$^\circ$C\\
		&&&unbehandelt&wärmebehandelt&&&&\\
		\hline \hline
		Aluminium&AlSi10Mg&2,57&230-270&&89,5&1.508,93&530&557\\
		Aluminium&AlSi7Mg0.6&2,67&250-255&&93,6&&&557\\
		\hline
		Edelstahl&1.4404&7,97&480-540&&60,2&4.991,35&850 (w)&1400\\
		Edelstahl&1.4542&7,79&861-861&1262-1262&110,5&2.559,27&550&1400\\
		Edelstahl&"CX"&7,69&840-8400&1650-1670&109,2&&&\\
		Edelstahl&1.4540&7,7&930-1025&1200-1250&120,8&&&\\
		\hline
		Inconel&IN 625&8,15&630-720&640-680&77,3&&950 (w)&1350\\
		Inconel&IN 718&8,15&&1140-1245&140,5&2.597,71&700&1260\\
		Inconel&IN 939&8,15&&1100-1130&135,0&&850&\\
		Inconel&"HX"&8,2&545-630&1200-1200&66,5&&&1355\\
		\hline
		Titan&Ti6Al4V&4,41&1120-1140&&254,0&3.085,12&>700 (w)&1630\\
		Titan&Ti6Al4V Grade 5&4,4&&970-1010&220,5&&870&1604\\
		Titan&Ti6Al4V ELI&4,41&&945-965&214,3&&982&2800\\
		\hline
	\end{tabular}
	\begin{flushright}
		\flushbottom{Quellen: \cite{eos, preise, T1.1, T1.3, T1.4, T2.1, T2.2, T2.3, T2.4, T3.1, T3.2, T3.3, T3.4, T3.5, T3.6, T3.7, T3.8}}
	\end{flushright}
\end{table}
(w) = Temperatur für Warmumformung
\end{landscape}