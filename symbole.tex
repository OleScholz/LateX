\chapter*{Symbolverzeichnis}						% Symbolverzeichnis
\addcontentsline{toc}{chapter}{Symbolverzeichnis}

\begin{longtable}[l]{llr}
\multicolumn{3}{l}{\textbf{Lateinische Buchstaben}}\\
$A$&Querschnittsfläche der Grid Fins&m$^2$\\
$C$&Kräfte- /Momentenbeiwert&-\\
$D$&Durchmesser&m\\
$F$&Kraft&N\\
$M$&Moment&Nm\\
$R$&Festigkeit&N/m$^2$\\
r&Radius&m\\
T&Periodendauer&s\\
U&Spannung&[V]\\
$\vec{U}$&Geschwinkigkeit des Fluids&m/s\\
$V$&Volumen&m$^3$\\
~&&\\
$a$&Abstand von Halterung A und B&m\\
$b$&Spannweite der Grid Fins&m\\
$d$&Wanddicke&m\\
$g$&Zellgröße, Abstand der Zellwände&m\\
$\vec{g}$&Erdbeschleunigung&m/s$^2$\\
$h$&Höhe der Grid Fins&m\\
$i$&Übersetzung&-\\
j&Anzahl der Schnittflächen bei Scherbelastung&-\\
$k$&Konstante&-\\
$m$&Masse&kg\\
$n$&Drehzahl&U/min\\
$s$&Sehnenlänge&m\\
$t$&Zeit nach der Trennung vom Flugzeug&s\\
\end{longtable}


\begin{longtable}[l]{llr}
	\multicolumn{3}{l}{\textbf{Griechische Buchstaben}}\\
	$\Phi$&Rollwinkel&rad\\
	$\Lambda$&Klappwinkel, (mit Index) Pfeilungswinkel&rad\\
	$\alpha$&Anstellwinkel des Grid Fins zur Anströmung&rad\\
	$\beta$&Neigungswinkel des Flugkörper zur Strömung&rad\\
	$\delta$&Steuerwinkel&rad\\
	$\eta$&Effizienz&-\\
	$\mu$&Reibungskoeffizient&-\\
	$\varphi$&Drehwinkel des Klappaktuators&-rad\\
	$\rho$&Dichte&kg/m$^3$\\
	$\sigma$&Normalspannung&N/m$^2$\\
	$\tau$&Schubspannung&N/m$^2$\\
\end{longtable}

\begin{longtable}[l]{ll}
	\multicolumn{2}{l}{\textbf{Indices}}\\
	$L$&Auftrieb\\
	$D$&Widerstand\\
	$E$&Einsatz\\
	$G$&Gitter\\
	$N$&Normal zur X-Achse\\
	$R$&Rahmen\\
	$Z$&Zelle\\
	$a$&Axial\\
	$b$&Körperfest\\
	$h$&In Höhenrichtung\\
	$m$&Auf das Steuergelenk bezogen\\
	$n$&Auf die Drehzahl bezogen\\
	$r$&Radial\\
	$s$&In Sehnenrichtung\\
	$\alpha$&Differnzialquotient über Anstellwinkel $\alpha$\\
	$\infty$&Zustand der Anströmung\\
	Konf&Konfiguration\\
	spez.&Spezifische (Festigkeit)\\
	zul.&Zulässige (Spannung)\\
	1, 2, 3&Grid Fin feste Koordinatenrichtungen\\
	$\zeta ,\ \eta ,\ \xi$&Grid Fin feste Koordinatenrichtungen ($45^\circ$ zu 1, 2, 3 gedreht)\\
\end{longtable}

\begin{longtable}[l]{ll}
	\multicolumn{2}{l}{\textbf{Abkürzungen}}\\
	FADS&Flush Air Data System\\
	COTS&Commercial off-the-shelf\\
	D1, R1, D2, R2&Positionen der Grid Fins (vgl. Abbildung \ref{abb_winkel})\\
	GAIA&German Association for Intercontinental Astronautics e.V.\\
	LEO&Low Earth Orbit\\
	LOX&Liquid Oxygen\\
	Max Q&Zeitpunkt des maximalen Staudrucks\\
	MOAB&Massive Ordiance Air Blast\\
	RCS&Reaction Control System\\
	RP-1&Rocket Propellant 1\\
	SP&Schwerpunkt\\
\end{longtable}