\chapter*{Symbolverzeichnis}						% Symbolverzeichnis
\addcontentsline{toc}{chapter}{Symbolverzeichnis}

\begin{longtable}[l]{llr}
\multicolumn{3}{l}{\textbf{Lateinische Buchstaben}}\\
$A$&Querschnittsfläche der Grid Fins&[m$^2$]\\
$C$&Kräfte- /Momentenbeiwert&[]\\
$D$&Durchmesser&[m]\\
$F$&Kraft&[N]\\
$M$&Moment&[Nm]\\
$R$&Festigkeit&[N/m$^2$]\\
~&Radius&[m]\\
T&Periodendauer&[s]\\
$U$&Geschwinkigkeit des Fluids&[m/s]\\
~&Spannung&[V]\\
$V$&Volumen&[m$^3$]\\
$a$&Abstand von Halterung A und B&[m]\\
$b$&Spannweite der Grid Fins&[m]\\
$d$&Wanddicke&[m]\\
$g$&Zellgröße, Abstand der Zellwände&[m]\\
~&Erdbeschleunigung&[m/s$^2$]\\
$h$&Höhe der Grid Fins&[m]\\
$i$&Übersetzung&[]\\
$k$&Konstante&~\\
$m$&Masse&[kg]\\
~&Anzahl der Schnittflächen bei Scherbelastung&[]\\
$n$&Drehzahl&[U/min]\\
$s$&Sehnenlänge&[m]\\
$t$&Zeit nach der Trennung vom Flugzeug&[s]\\
\end{longtable}


\begin{longtable}[l]{llr}
	\multicolumn{3}{l}{\textbf{Griechische Buchstaben}}\\
	$\Lambda$&Klappwinkel, (mit Index) Pfeilungswinkel&[rad]\\
	$\alpha$&Anstellwinkel des Grid Fins zur Anströmung&[rad]\\
	$\delta$&Steuerwinkel&[rad]\\
	$\eta$&Effizienz&[]\\
	$\lambda$&Rollwinkel&[rad]\\
	$\mu$&Reibungskoeffizient&[]\\
	$\phi$&Drehwinkel des Klappaktuators&[rad]\\
	$\rho$&Dichte&[kg/m$^3$]\\
	$\sigma$&Neigungswinkel des Flugkörper zur Strömung&[rad]\\
	~&Normalspannung&[N/m$^2$]\\
	$\tau$&Schubspannung&[N/m$^2$]\\
\end{longtable}

\begin{longtable}[l]{ll}
	\multicolumn{2}{l}{\textbf{Indices}}\\
	$A$&Auftrieb\\
	$D$&Widerstand\\
	$E$&Einsatz\\
	$G$&Gitter\\
	$N$&Normal zur X-Achse\\
	$R$&Rahmen\\
	$X$&In (negative) X-Richtung\\
	$Z$&Zelle\\
	$a$&Axial\\
	$b$&Körperfest\\
	$h$&In Höhenrichtung\\
	$m$&Auf das Steuergelenk bezogen\\
	$n$&Auf die Drehzahl bezogen\\
	$r$&Radial\\
	$s$&In Sehnenrichtung\\
	$\alpha$&Differnzialquotient über Anstellwinkel $\alpha$\\
	$\infty$&Zustand der Anströmung\\
	Konf&Konfiguration\\
	spez.&Spezifische (Festigkeit)\\
	zul.&Zulässige (Spannung)\\
	1, 2, 3&Grid Fin feste Koordinatenrichtungen\\
	$\zeta ,\ \eta ,\ \xi$&Grid Fin feste Koordinatenrichtungen ($45^\circ$ zu 1, 2, 3 gedreht)\\
\end{longtable}

\begin{longtable}[l]{ll}
	\multicolumn{2}{l}{\textbf{Abkürzungen}}\\
	AFDS&Air Flush Data System\\
	COTS&Commercial off-the-shelf\\
	D1, R1, D2, R2&Positionen der Grid Fins (vgl. Abbildung \ref{abb_winkel})\\
	GAIA&German Association for Intercontinental Astronautics e.V.\\
	LEO&Low Earth Orbit\\
	LOX&Liquid Oxygen\\
	Max Q&Zeitpunkt des maximalen Staudrucks\\
	MOAB&Massive Ordiance Air Blast\\
	RCS&Reaction Control System\\
	RP-1&Rocket Propellant 1\\
	SP&Schwerpunkt\\
\end{longtable}