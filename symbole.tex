\chapter*{Symbolverzeichnis}						% Symbolverzeichnis
\addcontentsline{toc}{chapter}{Symbolverzeichnis}

\begin{longtable}[l]{ll}
\multicolumn{2}{l}{\textbf{Lateinische Buchstaben}}\\
$A$&Querschnittsfläche der Grid Fins\\
$C$&Kräfte- /Momentenbeiwert\\
$F$&Kraft\\
$M$&Moment\\
$R$&Festigkeit\\
$U$&Geschwinkigkeit des Fluids\\
$V$&Volumen\\
$b$&Spannweite der Grid Fins\\
$g$&Zellgröße, Abstand der Zellwände; Erdbeschleunigung\\
$h$&Höhe der Grid Fins\\
$s$&Sehnenlänge\\
$t$&Zeit nach der Trennung vom Flugzeug\\
\end{longtable}


\begin{longtable}[l]{ll}
	\multicolumn{2}{l}{\textbf{Griechische Buchstaben}}\\
	$\Lambda$&Klappwinkel, Pfeilungswinkel (mit Index)\\
	$\alpha$&Anstellwinkel des Grid Fins zur Anströmung\\
	$\eta$&Steuerwinkel\\
	$\lambda$&Rollwinkel\\
	$\rho$&Dichte\\
	$\sigma$&Neigungswinkel des Flugkörper zur Strömung\\
	~&Spannung\\
\end{longtable}

\begin{longtable}[l]{ll}
	\multicolumn{2}{l}{\textbf{Indices}}\\
	$E$&Einsatz\\
	$G$&Gitter\\
	$Konf$&Konfiguration\\
	$N$&Normal zur X-Achse\\
	$R$&Rahmen\\
	$X$&In (negative) X-Richtung\\
	$Z$&Zelle\\
	$b$&Körperfest\\
	$h$&In Höhenrichtung\\
	$m$&Auf das Steuergelenk bezogen\\
	$s$&In Sehnenrichtung\\
	$\alpha$&Differnzialquotient über Anstellwinkel $\alpha$\\
	$\infty$&Zustand der Anströmung\\
\end{longtable}

\begin{longtable}[l]{ll}
	\multicolumn{2}{l}{\textbf{Abkürzungen}}\\
	COTS&Commercial off-the-shelf\\
	GAIA&German Association for Intercontinental Astronautics e.V.\\
	LEO&Low Earth Orbit\\
	LOX&Liquid Oxygen\\
	MOAB&Massive Ordiance Air Blast\\
	RCS&Reaction Control System\\
	RP-1&Rocket Propellant 1\\
	SP&Schwerpunkt\\
\end{longtable}