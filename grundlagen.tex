\chapter{Grundlagen}
\label{sec:grundlagen}

\section{Grid Fins als Steuerelement von Flugkörpern im Hyperschall}
\subsection{Aufbau}
Um Grid Fins und ihre Orientierung überhaupt beschreiben zu können werden zunächst einige Größe eingeführt.
In der simpelsten Konfiguration bestehen Grid Fins aus einem äußeren Rahmen, der die innere Struktur von sich kreuzenden planaren Flächen stützt. Dieser einfache Aufbau gewährt hohe Stabilität bei vergleichsweise geringem Gewicht \cite{zellform} und lässt sich mittels 5 Parameter, wie in Abbildung \ref{abb_parameter} zu sehen, beschreiben. Die Wanddicke $d$ kann sich für den Rahmen ($d_R$) von der für das Gitter ($d_G$) unterscheiden. Aber auch innerhalb dieser Einteilung kann der Wert variieren, so ist häufig die Wandstärke in der Nähe der Einspannung zu erhöhen, um die dort auftretenden höheren Beanspruchungen zu ertragen. Ein umrahmtes Segment des Gitters wird als Zelle bezeichnet und ihre Dimension kann mit die Zellgröße $g$ beschrieben werden. Die Ausmaße der Grid Fins wird maßgeblich durch die Spannweite $b$ und die Höhe $h$ bestimmt. Die Querschnittsfläche $A$ steht in der Ausgangsstellung senkrecht zur Anströmung und wird vom Rahmen begrenzt. Normal zu dieser Fläche steht die Sehne mit einer im Vergleich zur planaren Finne deutlich kürzeren Länge $s$.\\
\begin{figure}[h]
	\centering
	\includegraphics[width=0.6\textwidth]{Parameter.png}
	\caption{Aufbau eines einfachen Grid Fins}
	\label{abb_parameter}
\end{figure}\\
Grid Fins müssen nicht starr an einem Körper befestigt werden, sondern können um mehrere Achsen drehbar sein. Um sie für den Transport kompakt zu lagern, lassen sie sich an den Körper anlegen. Der Klappwinkel $\Lambda$ beschreibt den Ausschlag um eine den Körper an der Anbringung tangierende Achse. Ein Klappwinkel von $0^\circ$ entspricht hierbei den normalen in den Strömung ragenden Zustand und $90^\circ$ den eingeklappten. Zur Steuerung lassen sich die Grid Fins um eine Achse, die senkrecht aus dem Körper durch die Mitte des Grid Fins zeigt, verstellen. Ein Steuerwinkel von $\eta = 0^\circ$ ist auch hier wieder die Ausgangsstellung, die Sehne ist parallel zur Strömung. Bei $\eta = 90^\circ$ würde also die Seitenkante zur Anströmung zeigen, die Querschnittsfläche $A$, also das Gitter, wird nicht durchströmt.
\begin{figure}[h]
	\centering
	\includegraphics[width=0.8\textwidth]{Winkel.png}
	\caption{Winkel zur Beschreibung der Orientierung der Grid Fins zum Körper\\a) Steuerwinkel, b) Klappwinkel, c) Drehwinkel, d) Neigungswinkel des Körpers zur Anströmung}
	\label{abb_winkel}
\end{figure}\\
Um die Aerodynamik zu untersuchen reichen diese Winkel nicht aus, da die Anströmung nicht parallel zur Rakete liegen muss. Der Neigungswinkel des gesamten Moduls zur Antrömung $\sigma$ setzt sich unter realen Bedingungen aus dem Schiebewinkel und dem Bahnneigungswinkel, unter Vernachlässigung des Windes, zusammen. Für die in dieser Arbeit durchgeführten Untersuchungen ist eine solche Aufteilung aber irrelevant. Damit aber keine Informationen und somit Genauigkeit verloren geht, wird stattdessen die Orientierung der Grid Fins auf dem Umfang betrachtet. Verwendet wird hier eine Anordnung von vier gleichmäßig verteilten Steuerelementen. Das Koordinatensystem ist so definiert, dass es seinen Ursprung genau in der Mitte dieser Konfiguration hat und die positive X-Achse zur Spitze des Flugkörpers, also entgegen der Anströmung, zeigt. Bei $\sigma \neq 0$ zeigt auch die Y-Achse einem Anteil der Strömung entgegen. Die Z-Achse ist folglich nach den Rechtssystem orthogonal zu den anderen beiden ausgerichtet. Um nun die Orientierung der Grid Fins um die X-Achse herum beschreiben zu können wird der Rollwinkel $\lambda$ eingeführt. Wenn eine '+'-Konfiguration vorliegt, befinden sich die einzelnen Finnen auf den Koordinatenachsen (X, Y) und der Rollwinkel ist gleich null. Im Gegensatz dazu bei der 'x'-Konfiguration sind sie um einen Winkel von $\lambda = 45^\circ$ verdreht. Der Anstellwinkel $\alpha$, den ein einzelner Grid Fin erfährt, lässt sich aus dem Anstellwinkel des Körpers und, in Abhängigkeit vom Rollwinkel und welcher der Finnen überhaupt betrachtet wird, aus dem Klapp- und Steuerwinkel bestimmen.
\subsection{Strömung durch Grid Fins}\label{stroe}
Um die Eigenschaften von Grid Fins analysieren zu können, ist es nötig die zugrunde liegenden strömungsmechanische Vorgänge zu verstehen. Dazu werden in diesem Abschnitt das Verhalten der Strömung im Unter-, Überschall und besonders auch im transsonischen Bereich mit Schwerpunkt auf Verdichtungsstöße besprochen.
Bei niedrigen Strömungsgeschwindigkeiten im Unterschall haben Grid Fins auf Grund ihrer geringen Dicke keinen großen Einfluss auf das Fluid, welches nahezu ungestört durch das Gitter fließen kann \cite{sb-sharp}. Mit steigenden Machzahlen tritt aber zunehmend der Effekt auf, dass die Strömung um die stumpfe Vorderkante des Gitters in die Zelle hinein expandiert. Zusammen mit der Grenzschichtbildung an den Zellwänden, die effektiv zu Verengung der durchströmten Fläche führt, wir die Strömung innerhalb der Zellen auf Geschwindigkeiten beschleunigt, die über der der Anströmung liegen \cite{synopsis}.
\begin{figure}[h]
	\centering
	\includegraphics[width=0.7\textwidth]{swallow.png}
	\caption{Stoßsystem einer Zelle aus \cite{synopsis}}
\end{figure}\\

Der transsonische Bereich wird ab einer Anströmungsmachzahl von circa $Ma_\infty=0,8$ erreicht \cite{machgrenzen} und ist für die Aerodynamik der Grid Fins eine sehr kritische Problematik. Sobald die Strömung innerhalb des Gitters eine Machzahl von $1$ überschreitet, kommt es zu einem Verdichtungsstoß am Ausgang der Zellen, der mit steigender Machzahl an Stärke zunimmt. Dieser führt zu einer Drosselung der Strömung, was den Effekt hat, dass ein Teil der Strömung verdrängt wird und sich stattdessen um den Grid Fin herum bewegt. Steigt nun auch $Ma_\infty$ über $1$ löst sich der Stoß von den Gitterwänden und verbindet sich zu einer unregelmäßigen 3D-Struktur in der Abströmung \cite{stroemung}. Wächst $Ma_\infty$ weiter an, so kommt es zu einem Verdichtungsstoß vor dem Grid Fin. Dies führt dazu, dass innerhalb der Zellen keine Drosselung mehr vorliegt \cite{stroemung}, stattdessen wird die Strömung schon durch den Stoß vor dem Gitter um dieses herum verdrängt \cite{synopsis}. Von den Vorderkanten gehen Schockwellen aus, die auf benachbarte Wände treffen und von ihnen reflektiert werden \cite{synopsis}. Steigt die Machzahl weiter an, so befinden sich diese Wellen auf steileren Bahnen bis sie gar nicht mehr auf die anderen Wände treffen. Des weiteren nähert sich der Verdichtungsstoß vor dem Grid Fin diesem immer weiter mit größer werdenden Strömungsgeschwindigkeiten an, bis es abgesehen von der direkten Umgebung der Wände gar nicht mehr zum Stoß kommt. Die einzelnen Zellen fungieren nun als Überschalldüse \cite{stroemung}, sodass die Strömung in den meisten Bereichen nicht mehr auf den Unterschall abgebremst wird. Der Stoß wurde vom Gitter "verschluckt".
\\~\\
Als ein besonderer Bereich ist noch die Ansatzregion zu betrachten, in der der Grid Fin an der Rakete angebracht ist. Schnittstellen von Wänden stellen ein erhöhtes Potenzial für blockierte Strömung dar. In der Ansatzregion befinden sich nicht nur vielen von diesen Schnittstellen, sondern auch die Wanddicke ist hier meistens am größten. Dies in Kombination mit einer schon durch die Grenzschichtwirkung des Körpers verzögerte Strömung, führt zu einer relativ großen Region verlangsamter Strömung oder gar Rückströmung, die mit der Machzahl an Größe gewinnt \cite{stroemung}. Bei einer Machzahl von ungefähr $Ma_\infty = 2$ erreicht diese jedoch ein Maximum, da die Strömung bei weiter steigenden Geschwindigkeiten von der umgebenden mitgerissen wird und diese Region somit wieder an Größe und Bedeutung verliert \cite{stroemung}.

Es ist nun also hervorzuheben, dass Grid Fins weder im Unterschall noch im hohen Überschall übermäßig starke Störungen der Strömung bewirken. Im transsonischen Bereich jedoch kommt es zu massiven Verdichtungsstößen, die zu einer starken Drosselung des Fluids führen.
\subsection{Aerodynamische Beiwerte und Vergleich zu planaren Finnen}
Nachdem nun die zugrunde liegende Strömung verstanden ist und Größen zur Beschreibung von Grid Fins etabliert wurden, werden nun die aerodynamischen Kräfte beschrieben. Hierbei wird der Vergleich zu den konventionelle planaren Finnen gezogen.
\begin{figure}[h]
	\centering
	\includegraphics[width=0.6\textwidth]{kraefte.png}
	\caption{Kräfte und Momente am Grid Fin}
\end{figure}\\
Relevant sind zum einen die Kräfte, die normal zur X-Achse, also in der X-Y-Ebene, $F_N$ liegen, die in die negative X-Richtung zeigen $F_X$ und das Moment $M_m$, dass um die Achse in der die Steuerfläche steuerbar gelagert ist.
\subsubsection{Gelenkmoment}
Ein großer Vorteil von Grid Fins ist ihr geringes Moment um das Steuergelenk, welches den Einsatz von kleineren, weniger leistungsstarken Aktuatoren ermöglicht. Was wiederum eine Einsparung an Gewicht und Kosten mit sich bringt. Der Grund für das niedrige Moment ist hauptsächlich die im Vergleich zur planaren Finne deutlich kürzere Sehne, die der Luftkraft nur einen kleinen Hebelarm bietet. Der Druckpunkt befindet sich schon bei niedrigen Machzahlen in der Nähe der Mitte der Sehne, die die Achse ist, um die der Grid Fin gedreht wird, und wandert mit steigender Machzahl wenn auch nur leicht weiter Richtung 50\% der Sehnenlänge \cite{vergleichPlanarNATO}. Dies führt dazu, dass das Gelenkmomentbeiwert $C_m$ mit steigender Machzahl abnimmt. Auch für Variationen des Anstellwinkels beleibt der Beiwert durchgehend auf einem niedrigen Niveau, deutlich unter dem seines planaren Gegenstücks \cite{vergleichPlanarNATO}. Es sei hier jedoch anzumerken, dass es möglich ist eine planare Steuerfläche mit einem geringeren Moment zu erhalten, indem die Gelenkachse durch den Druckpunkt gelegt wird. Durch die große Druckpunktwanderung ist dies aber nur für einen kleinen vorher gewählten Machzahlengebiet dem Grid Fin überlegen, der über einen großen Geschwindigkeitsgebiet konstant gute Charakteristiken bietet.
\begin{figure}[h]
	\centering
	\includegraphics[width=0.8\textwidth]{gelenkmoment.png}
	\caption{Gelenkmoment über Anstellwinkel bei a) \ensuremath{Ma=0,5}, b) $Ma=2,5$ aus \cite{vergleichPlanar}}
\end{figure}\\
\subsubsection{Normalkraft}
Die Normalkrafterzeugung ist ausschlaggebend für die Stabilität und Steuerbarkeit eines Flugkörpers. Die Steigung der Normalkraftkoeffizient über den Anstellwinkel $C_{N\alpha}$ bei einem Anstellwinkel von $\alpha = 0$ ist in Abbildung \ref{abb_bucket} zu sehen. Wie in Abschnitt \ref{stroe} beschrieben, führt die Drosselung im transsonischen Bereich dazu, dass die Strömung um den Grid Fin herum verdrängt wird. Dadurch büßt er einen nicht vernachlässigbaren Teil seiner Fähigkeit Normalkraft zu erzeugen ein. Dieser Effekt ist genau gegensätzlich zu konventionalen planaren Finnen, die im Transschall ihren maximalen Normakraftkoeffizienten $C_N$ erreichen \cite{synopsis}.
\begin{figure}[h]
	\centering
	\includegraphics[width=0.6\textwidth]{bucket.png}
	\caption{Normalkraftsbeiwertgradient bei $\alpha = 0$ über die Machzahl aus \cite{synopsis}}
	\label{abb_bucket}
\end{figure}\\
Während vergleichbare konventionelle Finnen im Unterschall und niedrigen Überschall ähnlich hohe Normalkräfte erzeugen können, werden sie im hohen Machbereich von Grid Fins übertroffen. Schon ab $Ma=2,5$ kann die Normalkraft das 1,5-fache betragen und dieser Wert steigt mit der Machzahl nur noch weiter an \cite{synopsis,vergleichPlanarNATO}.
\subsubsection{Axialkraft}
Die Axialkraft wird häufig als der größte Nachteil angesehen, auch wenn der sich für spezielle Anwendungen als "drag brake" nutzen lässt. Wie schon schon im Abschnitt \ref{stroe} erwähnt, wird die Strömung bei niedrigen Geschwindigkeiten nicht stark gestört, folglich kommt es auch nicht zu großen Axialkräften. Im transsonischen Bereich steigt der Beiwert durch die Drosselung der Verdichtungsstöße rasant an und erreicht bei einer Machzahl knapp unter 1 sein Maximum \cite{solver}. Danach nimmt der Wert wieder ab, bleibt aber beim drei- bis vierfachen des Koeffizienten konventioneller Finnen.
\subsubsection{Stabilität}
\subsubsection{Anstellwinkelcharakteristika}
\subsection{Bisherige Implementierung/ Grid Fin Varianten}
\subsubsection{Pfeilung}
Konfiguration\\
Gitter\\
Lokal\\
\subsubsection{Krümmung}
\subsubsection{Kanten}
\section{Wiedereintrittsbedingungen}

\section{Das Air-Launchsystem Valkyrie}

